\documentclass[12pt]{article}
\usepackage{zed-csp}
\usepackage[top=2.5cm, bottom=2.5cm, left=3cm, right=3cm]{geometry}
\usepackage{graphicx}
\begin{document}

\begin{Huge}
\begin{center}
\begin{normalsize}
\textbf{MAKERERE \includegraphics[scale=0.5]{logo} UNIVERSITY }\\


\textbf{FACULTY OF COMPUTING AND INFORMATICS TECHNOLOGY} \\
\textbf{SCHOOL OF COMPUTING AND INFORMATICS TECHNOLOGY} \\
\textbf{DEPARTMENT OF COMPUTER SCIENCE} \\
\textbf{BACHELOR OF SCIENCE IN COMPUTER SCIENCE} \\
\textbf{YEAR 2} \\
\textbf{BIT 2207 RESEARCH METHODOLOGY} \\
\textbf{Course Work: Literature review}\\
\end{normalsize}
\end{center}
\end{Huge}

\begin{center}
\begin{tabular}{|l|l|l|c|}
\hline NAME  & REG NO & STD NO \\\hline

ABILA Raphael& 16/U/2673/PS & 216006923 \\\hline

\end{tabular}
\paragraph{•}
Lecturer: ERNEST MWEBAZE \\
\paragraph{•}
9th March 2018

\end{center}

\newpage
\title{}\textbf{LITERATURE REVIEW ON ANDROID APPLICATION DEVELOPED ON ECLIPSE SOFTWARE.} 


\section{INTRODUCTION}

\paragraph{•}Now is an exciting time for mobile developers. Android \cite{kanjilal} is an open source architecture that includes the operating system, middleware, and its key applications along with a set of API libraries for writing mobile applications that can shape the look, feel, and function of mobile handsets. Mobile developers can now expand into the Android platform to enhance existing products. Without any artificial barriers, Android developers write applications that take full advantage of increasingly powerful mobile hardware. Mobile applications are a rapidly growing segment of the global mobile market. In this paper, we discuss on Android mobile platform for the mobile application development, layered approach for android. Google released Android which is an open-source mobile phone operating system which is Linux-based. Android becomes the most widely used OS on mobile phones.  Android\cite{Android} is mobile operating systems designed for increasingly powerful mobile hardware.   

\subsection{Platform overview}

\paragraph{•}Android is a software stack which is for only mobile devices. The Android SDK provides the tools. Android has a multitude of platform specific features that we need to be aware of as Titanium developers. Everything from its user interface components to its low level interfaces for Services and Intents make Android stand apart from other mobile operating systems. While Titanium's Javascript API does the lion's share of the work in terms of abstracting away these details, it's still very important to understand them in order to build high quality apps. The following content in this section will address each of the most important features of the Android operating system, as well as discuss how they are handled from the perspective of the Titanium API \cite{Kevin}. 
\subsection{Fundamentals}

\paragraph{•}Android applications are written in Java programming language. They are not executed using the standard Java Virtual Machine (JVM). \cite{arthur} Google has created a custom VM called Dalvik which is responsible for converting and executing Java byte code. All custom Java classes must be converted into a Dalvik compatible instruction set before being executed into an Android operating system.

\subsection{ Development}
\paragraph{•}The Android SDK provides set of application programming interfaces (APIs). Android applications can share data among one another and also access shared resources on the system securely.




\begin{thebibliography}{9}
\bibitem{kanjilal} Joydip Kanjilal. \textit{Android},Internet: https://www.developer.com/ws/android/understanding-the-android-platform-architecture.html , March 4, 2016 [March 9, 2018].

\bibitem{Kevin}  Kevin Whinnery. \textit{Android Platform},Internet:https://wiki.appcelerator.org/display/guides2/Android+Platform+Overview,Oct 31, 2013[March 9, 2018].

\bibitem{Android} Android Developer. \textit{Android Worlds most popular mobile platform},Internet:https://developer.android.com/about/index.html ,[March 9, 2018].

\bibitem{arthur} Arthur Morgan. \textit{CommonsWare},Internet:https://commonsware.com/blog/Articles/what-is-dalvik.html ,Aug 30, 2010[March 9, 2018].

\end{thebibliography}
\end{document}